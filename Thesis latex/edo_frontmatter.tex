% Auto-generated from PDF text; edit as needed.
\begin{ChineseBlock}
\chapter*{摘要}
\addcontentsline{toc}{chapter}{摘要}
本论文探讨了人工智能(AI)如何重塑产品经理(PM)的角色,特别关注其在核心产品管理活动中的整合方式以及由此产生的伦理影响。研究聚焦于三个核心问题:(1)随着人工智能的整合,产品经理的角色如何演变?(2)如何将人工智能战略性地整合到产品管理的核心活动中,以增强决策能力、团队协作和产品开发流程?(3)人工智能如何影响产品经理的伦理决策过程,特别是在客户信任、偏见与透明度等方面?为深入探讨上述问题,本研究开展了一项结构化问卷调查,样本涵盖来自中国(n=37)和美国(n=37)的共计 74 名产品管理专业人士。问卷结合了李克特量表题项与开放性问题,旨在提供关于产品管理中人工智能应用演变的定量与定性洞见。研究得出了三个主要发现。第一,产品经理的角色正在发生重大转变:人工智能正在将 PM 的关注点从传统的协调与执行,转向以数据为驱动的战略制定、技术素养以及跨学科领导力。第二,人工智能在产品管理各项活动中(尤其是决策、协作与开发流程)实现的战略性整合,在设定清晰目标的前提下,能够显著提升效率、团队一致性以及产品迭代速度。第三,人工智能的引入提升了产品经理的伦理责任,他们愈发需要管理算法的公平性、确保 AI 功能的透明性,并通过原则性的监督机制来维护用户信任。这些发现共同揭示了人工智能在产品管理领域所带来的挑战与机遇。人工智能并未取代产品经理,而是作为催化剂与协作者存在,一方面增强了该职业的价值,另一方面也对能力、判断力与责任提出了更高要求。

关键词:人工智能(AI)、产品管理、人机协作、伦理决策、中美比较

\end{ChineseBlock}

\chapter*{Abstract}
\addcontentsline{toc}{chapter}{Abstract}
This thesis investigates how artificial intelligence (AI) is reshaping the role of product managers (PMs), with particular attention to its integration into core product management activities and its ethical implications. The study addresses three central research questions: (1) How is the role of the product manager evolving with the integration of AI? (2) How can AI be strategically integrated into the core activities of product management to enhance decision-making, team collaboration, and product development processes? (3) How does AI impact the ethical decision-making processes of product managers, especially in relation to customer trust, bias, and transparency? To explore these questions, a structured survey was conducted with a cross-national sample of 74 product management professionals, evenly divided between China (n=37) and the United States (n=37). The survey combined Likert-scale items and open-ended responses to provide both quantitative and qualitative insights into the evolving practices of AI adoption in product management. The analysis yields three major findings. First, the role of the product manager is undergoing a significant transformation: AI is shifting the PM’s focus from coordination and execution toward data-driven strategy, technical fluency, and cross-disciplinary leadership. Second, the strategic integration of AI across product management activities—particularly decision-making, collaboration, and development workflows— yields measurable improvements in efficiency, team alignment, and product iteration speed when applied with clear goals. Third, AI’s presence elevates the ethical responsibilities of PMs, who are increasingly called upon to manage algorithmic fairness, ensure transparency in AI-enabled features, and safeguard user trust through principled oversight. Together, these findings illuminate the dual challenge and opportunity presented by AI in product management. Rather than displacing the PM, AI acts as both a catalyst and a companion, augmenting the profession while demanding new forms of competence, judgment, and accountability.

Keywords: Artificial Intelligence (AI), Product Management, Human-AI Collaboration, Ethical Decision-Making, Cross-Cultural Comparison (USA–China).

\tableofcontents
\cleardoublepage

