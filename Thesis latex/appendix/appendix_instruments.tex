\chapter{Survey Instruments and Rating Scales}
\label{app:instruments}

This appendix lists the questionnaires and rating scales administered during the session. Unless otherwise noted, Likert-type items used 7-point response scales.
\par\smallskip
\textbf{Language note.} All participant-facing instruments were administered in Simplified Chinese. For transparency, each item is first presented in its original Chinese wording (as used in the experiment), followed by an English translation (for reference only).

\section{Baseline (pre-task) measures}

\textbf{Prior-knowledge familiarity.}
\textit{Original (Chinese; administered in the experiment).}
\begin{ChineseBlock}
\noindent \textbf{先验知识熟悉度(术语熟悉度)。} 请评价你对每个术语的熟悉程度(4点量表:0 = 完全不熟悉,1 = 略微熟悉,2 = 比较熟悉,3 = 非常熟悉)。术语以英文专业词汇呈现:
\end{ChineseBlock}
\textit{English translation (for reference only).} Participants rated familiarity with each term on a 4-point scale (0 = Not at all familiar, 1 = Slightly familiar, 2 = Moderately familiar, 3 = Very familiar). Terms:

\begin{multicols}{2}
\begin{itemize}
  \item Heat flux
  \item Permeable pavement
  \item Reflective coating
  \item Cooling corridor
  \item Urban canyon
  \item Albedo
  \item Gene drive
  \item Base editing
  \item Prime editing
  \item Adeno-associated virus (AAV)
  \item Lipid nanoparticle
  \item Germ-line editing
  \item Wafer
  \item Lithography mask
  \item System-on-a-chip (SoC)
  \item Photolithography
  \item Legacy node
  \item Extreme ultraviolet lithography (EUV)
\end{itemize}
\end{multicols}

\textbf{Baseline trust in AI (3 items).}
\textit{Original (Chinese; administered in the experiment).}
\begin{ChineseBlock}
\noindent(1 = 强烈不同意,7 = 强烈同意)
\begin{itemize}
  \item 我通常信任人工智能工具生成的信息。
  \item 人工智能系统通常会提供准确且公平的结果。
  \item 我很乐意依靠人工智能来支持我的学习或工作任务。
\end{itemize}
\end{ChineseBlock}
\textit{English translation (for reference only).} (1 = Strongly disagree, 7 = Strongly agree)
\begin{itemize}
  \item I generally trust information generated by AI tools.
  \item AI systems usually provide accurate and fair results.
  \item I feel comfortable relying on AI to support my learning or work tasks.
\end{itemize}

\textbf{Baseline technology dependence (3 items).}
\textit{Original (Chinese; administered in the experiment).}
\begin{ChineseBlock}
\noindent(1 = 强烈不同意,7 = 强烈同意)
\begin{itemize}
  \item 我经常依赖数位工具来帮我记住或储存资讯。
  \item 当我不确定某事时,我的第一反应是询问AI工具或搜索引擎。
  \item 科技让我思考更有效率,相较于仅依靠记忆而言。
\end{itemize}
\end{ChineseBlock}
\textit{English translation (for reference only).} (1 = Strongly disagree, 7 = Strongly agree)
\begin{itemize}
  \item I often rely on digital tools to remember or store information for me.
  \item When I'm unsure about something, my first instinct is to ask an AI tool or search engine
  \item Technology helps me think more efficiently than relying only on my memory.
\end{itemize}

\textbf{Self-rated AI/digital skill (2 items).}
\textit{Original (Chinese; administered in the experiment).}
\begin{ChineseBlock}
\noindent(1 = 强烈不同意,7 = 强烈同意)
\begin{itemize}
  \item 我对使用人工智能驱动的应用程序或系统充满信心。
  \item 我通常学习如何快速使用新的数字工具。
\end{itemize}
\end{ChineseBlock}
\textit{English translation (for reference only).} (1 = Strongly disagree, 7 = Strongly agree)
\begin{itemize}
  \item I feel confident using AI-powered applications or systems.
  \item I usually learn how to use new digital tools quickly.
\end{itemize}

\section{Per-article (post-block) ratings}

After each article block, participants completed ratings to capture perceived load and experience. Cognitive-load items used explicit anchors; other items used 1--7 agreement scales unless noted.

\textbf{Cognitive load.}
\textit{Original (Chinese; administered in the experiment).}
\begin{ChineseBlock}
\begin{itemize}
  \item ``这项任务对脑力的要求有多高?''(1 = 一点要求都没有;7 = 要求极高)
  \item ``理解这篇文章的内容有多困难?''(1 = 很容易;7 = 非常困难)
\end{itemize}
\end{ChineseBlock}
\textit{English translation (for reference only).}
\begin{itemize}
  \item ``How mentally demanding was this task?'' (1 = Not at all demanding; 7 = Extremely demanding)
  \item ``How difficult was it to understand the content of this article?'' (1 = Very easy; 7 = Very difficult)
\end{itemize}

\textbf{AI experience (AI groups only).} (1 = Strongly disagree; 7 = Strongly agree)
\textit{Original (Chinese; administered in the experiment).}
\begin{ChineseBlock}
\noindent(1 = 强烈不同意;7 = 强烈同意)
\begin{itemize}
  \item ``AI 生成的摘要帮助我理解了这篇文章。''
  \item ``AI 生成的摘要帮助我记住了这篇文章。''
  \item ``人工智能的协助使任务变得更容易、更高效。''
  \item ``我对本文提供的人工智能帮助感到满意。''
  \item (可选)``我更喜欢在人工智能支持下完成此类任务,而不是没有人工智能支持。''
\end{itemize}
\end{ChineseBlock}
\textit{English translation (for reference only).} (1 = Strongly disagree; 7 = Strongly agree)
\begin{itemize}
  \item ``The AI-generated summary helped me understand the article.''
  \item ``The AI-generated summary helped me remember the article.''
  \item ``The AI assistance made the task easier and more efficient.''
  \item ``I am satisfied with the AI assistance provided for this article.''
  \item (Optional) ``I prefer completing this kind of task with AI support rather than without it.''
\end{itemize}

\textbf{Overall MCQ confidence.}
\textit{Original (Chinese; administered in the experiment).}
\begin{ChineseBlock}
\noindent ``总体而言,您对本文多项选择题的回答有多大信心?''(1 = 一点也不自信;7 = 无比自信)
\end{ChineseBlock}
\textit{English translation (for reference only).} ``Overall, how confident are you in your answers to the multiple-choice questions for this article?'' (1 = Not confident at all; 7 = Extremely confident)

\textbf{Post-block state trust and dependence (AI groups only).} These items were asked after the recall and MCQ tasks for each article, referring to the summary just used in that block (1 = Strongly disagree; 7 = Strongly agree).
\textit{Original (Chinese; administered in the experiment).}
\begin{ChineseBlock}
\noindent(1 = 强烈不同意;7 = 强烈同意)
\begin{itemize}
  \item 信任(明确的可信度表述):``你认为 AI 生成的摘要在帮助理解这篇文章方面有多可靠?''(\texttt{trust\_new})
  \item 依赖(明确的卸载表述):``在完成本文章块的记忆任务时,你在多大程度上依赖 AI 生成的摘要,而不是依赖你自己对文章的记忆?''(\texttt{dependence\_new})
\end{itemize}
\end{ChineseBlock}
\textit{English translation (for reference only).}
\begin{itemize}
  \item Trust (explicit credibility framing): ``How reliable did you consider the AI-generated summary to be for understanding the article?'' (\texttt{trust\_new})
  \item Dependence (explicit offloading framing): ``To what extent did you rely on the AI-generated summary rather than your own memory of the article?'' (\texttt{dependence\_new})
\end{itemize}

\section{End-of-session manipulation checks}

\textit{Original (Chinese; administered in the experiment).}
\begin{ChineseBlock}
\begin{itemize}
  \item ``你觉得摘要和文章整体上有多连贯和相互关联?''(1 = 非常碎片化;7 = 高度连贯)
  \item ``摘要在多大程度上帮助你看到文章中观点之间的关系?''(1 = 完全没有;7 = 非常多)
  \item 策略选择:``在阅读和回答问题时,我主要……''(将观点联系起来形成整体理解 vs.\ 记忆零散事实与细节)
\end{itemize}
\end{ChineseBlock}
\textit{English translation (for reference only).}
\begin{itemize}
  \item ``How coherent and interconnected did the summary and article feel overall?'' (1 = Very fragmented; 7 = Highly connected)
  \item ``To what extent did the summary help you see relationships between ideas in the article?'' (1 = Not at all; 7 = Very much)
  \item Strategy choice: ``When reading and answering questions, I mainly...'' (Connected ideas to form an overall understanding vs. Memorized separate facts and details)
\end{itemize}

\section{Session instructions (verbatim)}
\textit{Original (Chinese; administered in the experiment).}
\begin{ChineseBlock}
\begin{itemize}
  \item 您将阅读 3 篇不同的文章,每篇文章都有自己的记忆和理解测试。
  \item 尽力记住每篇文章中尽可能多的信息。
  \item 您的奖金奖励与您的测试准确性成正比——您提供的答案越正确,您的奖金就越高。
  \item 这项任务需要全神贯注;请避免分心、切换标签或匆忙。
\end{itemize}
\end{ChineseBlock}
\textit{English translation (for reference only).}
\begin{itemize}
  \item You will read 3 different articles, each followed by its own memory and comprehension test.
  \item Try your best to remember as much information as possible from each article.
  \item Your bonus reward is proportional to your test accuracy --- the more correct answers you provide, the higher your bonus.
  \item The task requires full attention; please avoid distractions, switching tabs, or rushing.
\end{itemize}

\textbf{Important information about the AI summaries (verbatim).}
\textit{Original (Chinese; administered in the experiment).}
\begin{ChineseBlock}
\begin{itemize}
  \item 根据您的情况,您可能会看到最多三个AI生成的摘要:一个在文章之前,一个在阅读过程中(您可以随时打开和关闭),以及一个在文章之后。
  \item AI 通常很有帮助,摘要大多数时候是准确的。
  \item 然而,AI 仍可能包含小错误或遗漏。
  \item 摘要旨在帮助您——而不是替代仔细阅读完整文章。
  \item 有些测试问题只能使用完整文章中的细节来回答。
  \item 请仔细阅读所有内容以最大化您的表现。
\end{itemize}
\end{ChineseBlock}
\textit{English translation (for reference only).}
\begin{itemize}
  \item Depending on your condition, you may see up to three AI-generated summaries: one before the article, one during the reading (which you can open and close at any time), and one after the article.
  \item AI is generally helpful, and summaries are accurate most of the time.
  \item However, AI can still contain minor mistakes or omissions.
  \item The summaries are meant to assist you --- not replace careful reading of the full article.
  \item Some test questions can only be answered using details from the full articles.
  \item Please read everything carefully to maximize your performance.
\end{itemize}

\section{Free recall task instructions (verbatim)}
\textbf{Free Recall (5 minutes).} Write everything you remember from the article in short, idea-based sentences.
\par\smallskip
\textit{Original (Chinese; administered in the experiment).}
\begin{ChineseBlock}
\noindent \textbf{自由回忆(5分钟)。} 请用简短、以观点为单位的句子写下你从文章中记住的所有内容。
\begin{itemize}
  \item 使用模式:``A $\rightarrow$ B 因为 C''(原因 $\rightarrow$ 结果 $\rightarrow$ 理由)或 ``X 导致 Y,因为 Z''。
  \item 目标写 8--12 句(用于评分的最大句数 = 10)。
  \item 关注具体观点与它们之间的联系,而不是泛泛的总体印象。
  \item 不要复制/粘贴;请用自己的话表达。
  \item 你有 5:00 分钟。计时器会显示剩余时间。
  \item 为确保你阅读了说明,开始按钮将在 30 秒后解锁。(实验中另设置了继续按钮的最短锁定时间。)
\end{itemize}
\end{ChineseBlock}
\textit{English translation (for reference only).}
\begin{itemize}
  \item Use the pattern: ``A $\rightarrow$ B because C'' (cause $\rightarrow$ effect $\rightarrow$ reason) or ``X leads to Y due to Z''.
  \item Aim for 8--12 sentences (max accepted for scoring = 10).
  \item Focus on specific ideas and links, not general impressions.
  \item No copy/paste; write in your own words.
  \item You have 5:00 minutes. A timer shows remaining time.
  \item The start button is unlocked after 30 seconds to ensure instructions are read.
\end{itemize}

\section{Consent text (verbatim)}
\textit{Original (Chinese; administered in the experiment).}
\begin{ChineseBlock}
您被邀请参加一项研究,探索不同的文本格式如何影响记忆和理解。今天的实验大约需要 90 分钟。您将阅读简短的科学文章、回答问题并评估您的经验。
\begin{itemize}
  \item 您的参与是自愿的。
  \item 您可以随时退出而不会受到处罚。
  \item 不会收集任何个人识别信息;数据将被匿名存储。
  \item 您必须年满 18 岁并使用笔记本电脑或台式电脑(不得使用手机)。
\end{itemize}
单击 ``我同意并继续'',即表示您确认已阅读此信息并同意参与。
\end{ChineseBlock}
\textit{English translation (for reference only).} You are invited to participate in a study exploring how different text formats influence memory and comprehension. The experiment takes approximately 90 minutes. You read short scientific articles, answer questions, and evaluate your experience.
\begin{itemize}
  \item Your participation is voluntary.
  \item You may withdraw at any time without penalty.
  \item No personal identifying information will be collected; data will be stored anonymously.
  \item You must be 18 years or older and use a laptop or desktop computer (no phones).
\end{itemize}
By clicking ``I Agree and Continue'', you confirm that you have read this information and consent to participate.

\section{Debrief screen (verbatim)}
The debrief screen thanked participants, displayed their participant ID, stated that responses help understand how AI influences learning/recall/comprehension, and instructed participants that they could close the window.
\par\smallskip
\textit{Chinese translation (for reference).}
\begin{ChineseBlock}
致谢页面感谢参与者,显示其参与者ID,并说明这些回答将用于理解 AI 如何影响学习/回忆/理解,同时提示参与者可以关闭窗口。
\end{ChineseBlock}
